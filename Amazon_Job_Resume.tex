% resume.tex
% vim:set ft=tex spell:

\documentclass[10pt,letterpaper]{article}
\usepackage[letterpaper,margin=0.40in]{geometry}
\usepackage[utf8]{inputenc}
\usepackage{mdwlist}
\usepackage[T1]{fontenc}
\usepackage{textcomp}
\usepackage{tgpagella}
\pagestyle{empty}
\setlength{\tabcolsep}{0em}
\usepackage[colorlinks = true,
            linkcolor = blue,
            urlcolor  = blue,
            citecolor = blue,
            anchorcolor = blue]{hyperref}

% indentsection style, used for sections that aren't already in lists
% that need indentation to the level of all text in the document
\newenvironment{indentsection}[1]%
{\begin{list}{}%
	{\setlength{\leftmargin}{#1}}%
	\item[]%
}
{\end{list}}

% opposite of above; bump a section back toward the left margin
\newenvironment{unindentsection}[1]%
{\begin{list}{}%
	{\setlength{\leftmargin}{-0.5#1}}%
	\item[]%
}
{\end{list}}

% format two pieces of text, one left aligned and one right aligned
\newcommand{\headerrow}[2]
{\begin{tabular*}{\linewidth}{l@{\extracolsep{\fill}}r}
	#1 &
	#2 \\
\end{tabular*}}

% make "C++" look pretty when used in text by touching up the plus signs
\newcommand{\CPP}
{C\nolinebreak[4]\hspace{-.05em}\raisebox{.22ex}{\footnotesize\bf ++}}

% and the actual content starts here
\begin{document}

\pagestyle{empty}

\begin{center}
{\LARGE \textbf{Arunav Sanyal}}

(236) 865-5547\ \
\ \ arunav.sanyal91@gmail.com
\end{center}


\subsection*{Education}

\begin{itemize}
	\parskip=0.1em

	\item 
	\headerrow
		{\textbf{Johns Hopkins University}}
		{\textbf{Baltimore, MD, USA}}
	\\
	\headerrow
		{\emph{M.S.E, Department of Computer Science}}
		{\emph{\textbf{2014 -- 2016}}}
	
	\item 
		\headerrow
			{\textbf{BITS Pilani University}}
			{\textbf{Zuarinagar, Goa, India}}
		\\
		\headerrow
			{\emph{B.E(Honors) Computer Science}}
			{\emph{\textbf{2009 -- 2013}}}

\end{itemize}

\hrule
\vspace{-0.4em}
\subsection*{Experience}

\begin{itemize}
	\parskip=0.1em
		\item
	\headerrow
	{\textbf{Amazon}}
	{\textbf{}}
	\\
	\headerrow
	{\emph{\textbf{Software Development Engineer III }}}
	{\emph{\textbf{July 2022 -- Present, Vancouver, British Columbia, Canada}}}
	\headerrow
	{\emph{\textbf{Software Development Engineer II }}}
	{\emph{\textbf{March 2021 -- June 2022  Vancouver, British Columbia, Canada} }}
	\headerrow
	{\emph{\textbf{Software Development Engineer II}}}
	{\emph{\textbf{August 2018 -- Feb 2021  Seattle, Washington, USA} }}
	\headerrow
	{\emph{\textbf{Software Development Engineer I}}}
	{\emph{\textbf{April 2016 -- July 2018  Seattle, Washington, USA} }}
	\begin{itemize*}
		\item I am currently a Software Development Engineer  III  and tech lead of AWS IoT  (Internet of Things) \href{https://docs.aws.amazon.com/iot/latest/developerguide/iot-rules.html}{Rules Engine team}. Previous I have worked in AWS RDS Aurora Serverless team and in Authenticity Team, Consumables Org. 
			\item My job duties include orchestrating/coordinating back end architectural design, product direction with stakeholders (technical and project management), operational excellence, software implementation, mentoring  and coaching junior engineers.  
			\item \textbf{A high level description of the teams I have worked on}
		\begin{itemize*}
			\item Rules Engine  is a high scale Java/Scala based EC2 micro services distributed system built using CFN/CDK on top of Akka Streams  that transforms messages with a custom interpreter and routes petabytes of data yearly into AWS Systems like S3, DynamoDB, Kinesis, SQS, SNS, Cloudwatch, ElasticSearch, Lambda. Timestream etc and  also custom systems like Kafka and generic HTTP Servers. 
			\item AWS Aurora is a complex orchestration/distributed system for managing/provisioning of Mysql and Postgres Databses on AWS Cloud. This involves working with AWS Step Functions and SWF to orchestrate database creation, patching and management for tens of thousands of SQL database instances. 
			\item Amazon Authenticity is a end to end tracking system for products in Amazon retail which involves multiple points of integration with the Retail Services and provides a mobile app to consumers to verify whether products are tampered with. 
		\end{itemize*}
		\item \textbf{Notable projects I worked on include}
		\begin{itemize*}
			\item I lead the design, implementation and testing effort for \href{https://aws.amazon.com/about-aws/whats-new/2022/12/aws-iot-core-rules-engine-google-protocol-buffer-messaging-format/}{Google protobuf format support} for AWS IoT core ingestion.
			\item I coordinated and led the effort for org wide improvements in high scale load testing (including writing the standard guidelines/template for it), investigating and implementing resiliency improvements for slow/misbehaving downstream ingestion which consume more client resources (threads/CPU etc) 
			\item I designed an implemented a "multi source" reactive stream system on Akka Streams to facilitate Rules Engine Data Plane ingestion on both Synchronous HTTP and Event based polling (SQS) as message ingestion sources. 
			\item  I led/designed/implemented low cost ingestion into AWS  IoT Core with \href{https://docs.aws.amazon.com/iot/latest/developerguide/iot-basic-ingest.html}{Basic Ingest}. The project involved simplifying ingestion by bypassing unnecessary services on the data path and reduction of per message costs by 70 percent (from 1 dollar to 30 cents per million messages)
			\item I implemented a "heat management system" for DB fleet orchestration, which involved collecting resource usage statistics from Database VMs (CPU, Memory, Bufferpool size etc) and then updated that state for an algorithm to decide when to provision more VMs to create Databases on. I also implemented the APIs for updating heat capacity and provisioning new EC2 AMIs and VMs, 
			\item I led a cost reduction effort in AWS IoT Core, finding opportunities in low cost logging, infra usage and internal messaging costs, which led to an infrastructure cost reduction by 55 percent.
			\item I implemented the firecracker File system image and EC2 AMI baking system for  \href{https://aws.amazon.com/about-aws/whats-new/2020/12/introducing-the-next-version-of-amazon-aurora-serverless-in-preview/}{RDS Aurora Serverless V2}, a dynamically auto scaling managed relational database solution on which VMs on which databases were hosted. 
			\item  I was part of the team that implemented \href{https://brandservices.amazon.com/transparency}{Transparency} Android and IoS Apps, in which my notable contribution was device version management and communication to back end services from mobile apps. 
		\end{itemize*}
	\end{itemize*}
	\item
		\headerrow
			{\textbf{Informatica Business Solutions-HQ/ILabs}}
			{\textbf{}}
		\\
		\headerrow
			{\emph{\textbf{Software Developer Intern}}}
			{\emph{\textbf{May 2015 -- August 2015, Redwood City, California, USA}}}
		\headerrow
			{\emph{\textbf{Software Developer}}}
			{\emph{\textbf{January 2013 -- June 2014, Bangalore, Karnataka, India}}}
		\begin{itemize*}
			\item I implemented the auto provisioning framework, allowing Informatica compute nodes to dynamically scale in both Amazon EC2 Cloud  and premises hardware. 
			\item I worked on integrating Kerberos Authentication framework for the PowerCentre platform (a distributed server system in Informatica), where I extended HTTP curl client and Apache Tomcat Server used by Powercenter to be fully kerberos compliant. 
		\end{itemize*}
\end{itemize}

\hrule
\vspace{-0.4em}

\subsection*{Open Source}

\begin{itemize*}
	\item I was a former maintainer of \href{https://github.com/goktugyil/EZSwiftExtensions}{EZSwiftExtensions} - A standard library for the Swift Lang and IOS extensions.
	\item I extended \href{https://github.com/Khalian/curl}{cURL} to fully support Kerberos v5. 
	
\end{itemize*}


\hrule
\vspace{-0.4em}

\subsection*{Academic Projects}

\begin{itemize}

\parskip=0.1em
	\item
		\headerrow
			{\textbf{Modulo7 : A full stack Music Information Retrieval Engine} }
			{\textbf{Johns Hopkins University}}
		\\
		\headerrow
				{\emph{May 2015 -- Present}}
		
		\begin{itemize*}
			\item I created a full stack MIR Engine using Music Theory principles. This engine would be able to process music sources of different formats (mp3, midi, sheet music etc) and allow a querying system for users to acquire statistics and similarities on music theory principles \hspace*{3pt}
			\href{https://github.com/Khalian/Modulo7}{Source Code} \hspace*{3pt}
			\href{https://jscholarship.library.jhu.edu/bitstream/handle/1774.2/40683/SANYAL-THESIS-2016.pdf}{Thesis}
		\end{itemize*} 
		
\end{itemize}

\hrule
\vspace{-0.4em}

\subsection*{Core Technical Skills}

\begin{indentsection}{\parindent}
\hyphenpenalty=1000
\begin{description*}
	\item[Languages:]
	Java, Scala, Swift, Typescript, Bash/Csh Shell scripting, \LaTeX
	\item[Tools:]
	IntelliJ, Git, Vim
\end{description*}
\end{indentsection}

\end{document}